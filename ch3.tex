\title{Chapter 3}
\date{\today}
\author{Brett Mullins}
\documentclass[12pt]{article}

\usepackage{amsmath, amsthm, amsfonts, verbatim, amssymb}

% Theorems
%-----------------------------------------------------------------
\newtheorem{thm}{Theorem}[section]
\newtheorem{cor}[thm]{Corollary}
\newtheorem{lem}[thm]{Lemma}
\newtheorem{prop}[thm]{Proposition}
\theoremstyle{definition}
\newtheorem{defn}[thm]{Definition}
\theoremstyle{remark}
\newtheorem{rem}[thm]{Remark}
\newtheorem{prob}{Problem}[section]

% Shortcuts.
% One can define new commands to shorten frequently used
% constructions. As an example, this defines the R and Z used
% for the real and integer numbers.
%-----------------------------------------------------------------
\def\RR{\mathbb{R}}
\def\ZZ{\mathbb{Z}}
\def\QQ{\mathbb{Q}}
\def\PP{\mathcal{P}}
\def\la{\lambda}
\def\contra{\rightarrow \leftarrow}
\def\empty{\varnothing}

% Similarly, one can define commands that take arguments. In this
% example we define a command for the absolute value.
% -----------------------------------------------------------------
\newcommand{\abs}[1]{\left\vert#1\right\vert}
\begin{document}
\maketitle
%
% Problem 1
%

\section{}
%
% Problem 1(a)
%
Let $X$ denote the set of prizes and $b \in X$ be the best prize and $w \in X$ be the worst prize. Let $\delta_b$ denote the lottery that results in $b$ for sure. We shall assume that $\delta_b \succ \delta_w$.
\begin{thm}
  \label{vN-M1}
  For $\alpha, \beta \in [0,1]$, $\alpha \delta_b + (1 - \alpha) \delta_w \succ \beta \delta_b + (1 - \beta) \delta_w$ if and only if $\alpha > \beta$.
\end{thm}
\begin{proof}
  Let $\alpha, \beta \in [0,1]$.
  \begin{enumerate}
    \item Case 1: Suppose $\alpha = 1$.
    \begin{enumerate}
      \item $(\rightarrow)$ Suppose $\alpha \delta_b + (1 - \alpha) \delta_w \succ \beta \delta_b + (1 - \beta) \delta_w$. Then $\delta_b \succ \beta \delta_b + (1 - \beta)\delta_w$. For contradiction, suppose $\beta = 1$. Then $\delta_b \succ \delta_b$. $(\contra)$.
      \item $(\leftarrow)$ Suppose $\alpha > \beta$. Then $\beta < 1$. We know that $\delta_b \succ \delta_w$. By the Substitution Axiom, we have that
      \begin{align*}
        &          & (1 - \beta)\delta_b + (1-(1-\beta))\delta_b &\succ (1 - \beta)\delta_w + (1-(1-\beta))\delta_b \\
        & \implies & \delta_b &\succ \beta \delta_b + (1 - \beta)\delta_w.
      \end{align*}
    \end{enumerate}
    \item Case 2: $\alpha \in (0,1)$.
    \begin{enumerate}
      \item $(\leftarrow)$ Suppose $\alpha > \beta$. Since $\delta_b \succ \delta_w$, by the Substitution Axiom, we obtain
      \begin{align*}
        &          & \alpha \delta_b + (1-\alpha)\delta_w &\succ \alpha \delta_w + (1 - \alpha)\delta_w \\
        & \implies & \alpha \delta_b + (1-\alpha)\delta_w &\succ \beta \delta_w.
      \end{align*}

      Let $\gamma = \frac{\beta}{\alpha}$ and $r = \alpha \delta_b + (1 - \alpha) \delta_w$. Then we have that $r \succ \delta_w$. Applying the Substitution Axiom once more, we obtain
      \begin{align*}
        &          & (1 - \gamma) r + \gamma r &\succ (1 - \gamma) \delta_w + \gamma r \\
        & \implies & r &\succ \delta_w - \gamma \delta_w + \gamma (\alpha \delta_b + (1 - \alpha) \delta_w) \\
        & \implies & r &\succ \delta_w - \gamma \delta_w + \gamma \alpha \delta_b + \gamma \delta_w - \gamma \alpha \delta_w \\
        & \implies & r &\succ \delta_w + \beta \delta_b - \beta \delta_w \\
        & \implies & r &\succ \beta \delta_b + (1 - \beta) \delta_w.
      \end{align*}
      \item $(\rightarrow)$ Suppose $\alpha \delta_b + (1 - \alpha)\delta_w \succ \beta \delta_b + (1 - \beta) \delta_w$. For contradiction, suppose that $\beta \geq \alpha$.
      \begin{enumerate}
        \item Suppose $\beta = \alpha$. Then $\alpha \delta_b + (1 - \alpha)\delta_w \succ \alpha \delta_b + (1 - \alpha)\delta_w$, violating asymmetry. $(\contra)$.
        \item Then $\beta > \alpha$. By the cases above, we have that $\beta \delta_b + (1 - \beta) \delta_w \succ \alpha \delta_b + (1 - \alpha)\delta_w$, also violating assymetry. $(\contra)$.
      \end{enumerate}
    \end{enumerate}
  \end{enumerate}
\end{proof}
(A special thanks to Eric Pacuit for a clear explanation of what I was missing from the above proof.)
%
% Problem 1(b)
%
\begin{thm} \label{lemma 2}
  For any lottery $p \in P$, there exists $\alpha \in [0,1]$ such that $p \sim \alpha \delta_b + (1 - \alpha) \delta_w$.
\end{thm}
\begin{proof}
  Let $p \in P$.
  \begin{enumerate}
    \item Suppose $p \sim \delta_b$. Then $\alpha = 1$ satisfies the property. Similarly, for $p \sim \delta_w$, $\alpha = 0$ is sufficient.
    \item Suppose $p \in P \backslash \{ \delta_b, \delta_w \}$. Let us define $\alpha = \inf \{ \beta \mid \beta \delta_b + (1 - \beta) \delta_w \succ p \}$. We claim that $\alpha \delta_b + (1 - \alpha)\delta_w \sim p$.
    \begin{enumerate}
      \item Suppose $r = \alpha \delta_b + (1 - \alpha)\delta_w \succ p$. Then $r \succ p \succ \delta_w$. By the Archimedean Axiom, there exist $\gamma_1, \gamma_2 \in (0,1)$ such that $\gamma_1 r + (1 - \gamma_1) \delta_w \succ p \succ \gamma_2 r + (1 - \gamma_2) \delta_w$.
      Applying the Substitution Axiom, we have that $r \succ \gamma_1 r + (1 - \gamma_1) \delta_w$. Observe that
      \begin{align*}
        \gamma_1 r + (1 - \gamma_1) \delta_w &= \gamma_1 (\alpha \delta_b + (1 - \alpha)\delta_w) + \delta_w - \gamma_1 \delta_w \\
        &= \gamma_1 \alpha \delta_b + \gamma_1 (1 - \alpha)\delta_w + \delta_w - \gamma_1 \delta_w \\
        &= \gamma_1 \alpha \delta_b + \gamma_1 \delta_w - \gamma_1 \alpha \delta_w + \delta_w - \gamma_1 \delta_w \\
        &= \gamma_1 \alpha \delta_b - \gamma_1 \alpha \delta_w + \delta_w \\
        &= \gamma_1 \alpha \delta_b (1 - \gamma_1 \alpha)\delta_w.
      \end{align*}
      By applying (\ref{vN-M1}), $\alpha > \gamma_1 \alpha$, but $\gamma_1 \alpha \in \{ \beta \mid \beta \delta_b + (1 - \beta) \delta_w \succ p \}$. $(\contra)$.
      \item Suppose $p \succ r$. Then we have that $\delta_b \succ p \succ r$. Applying the Archimedean Axiom, there exists $\gamma \in (0,1)$ such that $p \succ \gamma \delta_b + (1 - \gamma) r$. Applying the Substitution Axiom, we find that $\gamma \delta_b + (1 - \gamma) r \succ r$. Observe that
      \begin{align*}
        \gamma \delta_b + (1 - \gamma) r &= \gamma \delta_b + r - \gamma r \\
        &= \gamma \delta_b + \alpha \delta_b + (1 - \alpha) \delta_w - \gamma (\alpha \delta_b + (1 - \alpha) \delta_w) \\
        &= \gamma \delta_b + \alpha \delta_b + \delta_w - \alpha \delta_w - \gamma \alpha \delta_b - \gamma \delta_w + \gamma \alpha \delta_w \\
        &= (\gamma + \alpha - \gamma \alpha) \delta_b + (1 - \alpha - \gamma + \gamma \alpha)\delta_w. \\
      \end{align*}
      We claim that $(\gamma + \alpha - \gamma \alpha) \in [0,1]$.
      \begin{enumerate}
        \item Suppose $(\gamma + \alpha - \gamma \alpha) > 1$. Then
        \begin{align*}
          &           & (\gamma + \alpha - \gamma \alpha) &> 1 \\
          & \implies  & (\gamma - \gamma \alpha) &> 1 - \alpha \\
          & \implies  & (1 - \alpha)\gamma &> 1 - \alpha \\
          & \implies  & \gamma &> 1.
        \end{align*}
        $(\contra)$.
        \item Suppose $(\gamma + \alpha - \gamma \alpha) < 0$. Then
        \begin{align*}
          &          & (\gamma + \alpha - \gamma \alpha) &< 0 \\
          & \implies & (\gamma - \gamma \alpha) &< -\alpha \\
          & \implies & (1 - \alpha)\gamma &< -\alpha \\
          & \implies & \gamma &< -\frac{\alpha}{1 - \alpha}.
        \end{align*}
        $(\contra)$.
      \end{enumerate}
      Applying (\ref{vN-M1}), $(\gamma + \alpha - \gamma \alpha) > \alpha$. Then $(\gamma + \alpha - \gamma \alpha) \not \in \{ \beta \mid \beta \delta_b + (1 - \beta) \delta_w \succ p \}$ but $(\gamma + \alpha - \gamma \alpha) > \alpha$.
      Hence, $\alpha$ is not the greatest lower bound. $(\contra)$.
    \end{enumerate}
  \end{enumerate}
\end{proof}
%
% Problem 1(c)
%
\begin{thm}\label{lemma 3}
  If $p \sim q$, $r \in P$, and $\alpha \in [0,1]$, then $\alpha p + (1 - \alpha)r \sim \alpha q + (1 - \alpha)r$.
\end{thm}
\noindent Let us prove some preliminary results to aid in the read ability of this proof. Two of these lemmas are presented earlier in the book without proof.
\begin{lem} \label{lem1c.1}
  If $p \sim q$, $r \succ p$, $q \succ s$, for $p, q, r, s \in P$, then $r \succ q$ and $p \succ s$.
\end{lem}
\begin{proof}
  Suppose $p \sim q$, $r \succ p$, $q \succ s$, for $p, q, r, s \in P$. Then $p \not \succ q$ and $q \not \succ p$.
  \begin{enumerate}
    \item Since $r \succ p$, by negative transitivity, either $r \succ q$ or $q \succ p$. As the latter implies a contradiction, the former obtains.
    \item Since $q \succ s$, by negative transitivity, either $q \succ p$ or $p \succ s$. As the former implies a contradiction, the latter obtains.
  \end{enumerate}
\end{proof}
\begin{lem} \label{lem1c.2}
  If $p \succ q$, $r \succ s$, and $\alpha \in [0,1]$, then $\alpha p + (1 - \alpha)r \succ \alpha q + (1 - \alpha)s$.
\end{lem}
\begin{proof}
  Suppose that $p \succ q$, $r \succ s$. Observe that if $\alpha = 1$ or $\alpha = 0$, then the result follows immediately. So we will suppose that $\alpha \in (0,1)$. By the Substitution Axiom, $\alpha p + (1 - \alpha)r \succ \alpha q + (1 - \alpha)r$ and $(1 - \alpha)r + \alpha q \succ ( 1 - \alpha )s + \alpha q$.
  Since $\succ$ is transitive, $\alpha p + (1 - \alpha)r \succ \alpha q + (1 - \alpha)s$.
\end{proof}
\begin{lem} \label{lem1c.3}
  If $p \sim q$ and $\alpha \in [0,1]$, then $\alpha p + (1 - \alpha)q \sim p$.
\end{lem}
\begin{proof}
  Suppose $p \sim q$ and $\alpha \in [0,1]$. Let $r = \alpha p + (1 - \alpha)q$. For contradiction, suppose either $r \succ p$ or $p \succ r$.
  \begin{enumerate}
    \item Suppose $r \succ p$. By (\ref{lem1c.1}), $r \succ q$. By (\ref{lem1c.2}),
    \begin{align*}
      &           & \alpha r + (1 - \alpha)r &\succ \alpha p + (1 - \alpha)q \\
      & \implies   & r &\succ r
    \end{align*}
    $(\contra)$.
    \item Suppose $p \succ r$. By (\ref{lem1c.1}), $q \succ r$. By (\ref{lem1c.2}),
    \begin{align*}
      &           & \alpha p + (1 - \alpha)q  &\succ \alpha r + (1 - \alpha)r \\
      & \implies   & r &\succ r
    \end{align*}
    $(\contra)$.
  \end{enumerate}
\end{proof}
\begin{proof}
  Suppose $p \sim q$, $r \in P$, and $\alpha \in [0,1]$. Observe that if $\alpha = 1$ or $\alpha = 0$, then the result follows immediately.
  We will suppose that $\alpha \in (0,1)$.
  \begin{enumerate}
    \item Suppose $p \sim r$. Then by (\ref{lem1c.3}) $\alpha p + (1 - \alpha)r \sim p$ and $\alpha p + (1 - \alpha)r \sim q$. Since $\sim$ is transitive, $\alpha p + (1- \alpha)r \sim \alpha q + (1 - \alpha)r$.
    \item Suppose $p \succ r$.
    \begin{enumerate}
      \item For contradiction, suppose $\alpha q + (1 - \alpha)r \succ \alpha p + (1 - \alpha)r$. By the Substitution Axiom, we have that $\alpha p + (1 - \alpha)r \succ r$.
      By (\ref{lem1c.2}), there exists $\beta \in [0,1]$ such that
      \begin{align*}
        &           & \alpha p + (1 - \alpha)r &\sim (1 - \beta)r + \beta[\alpha q + (1 - \alpha)r] \\
        & \implies  & \alpha p + (1 - \alpha)r &\sim \alpha \beta q + (1 - \alpha \beta)r.
      \end{align*}
      By (\ref{lem1c.1}), we have that $q \succ r$. Then, by the Substitution Axiom, $q \succ \beta q + (1 - \beta)r$. Then by (\ref{lem1c.1}), $p \succ \beta q + (1 - \beta)r$. By the Substitution Axiom,
      \begin{align*}
        &           & \alpha p + (1 - \alpha)r &\succ \alpha[\beta q + (1 - \beta)r] + (1 - \alpha)r \\
        & \implies  & \alpha p + (1 - \alpha)r &\succ \alpha \beta q + (1 - \alpha \beta)r.
      \end{align*}
      $(\contra)$
      \item For the case in which $\alpha p + (1 - \alpha)r \succ \alpha q + (1 - \alpha)r$, switch the roles of $p, q$ and the argument follows.
      \end{enumerate}
    \item The case for $r \succ p$ follows similarly.
  \end{enumerate}
\end{proof}
%
% Problem 1(d)
%
\begin{thm}
  Let $u(x) \in [0,1]$ be defined as $\delta_x \sim u(x)\delta_b + (1 - u(x))\delta_w$. For $u:X \rightarrow \RR$, for $p \in P$, $$p \sim \sum_{x \in supp(p)} u(x)p(x) \delta_b + \left( 1 - \sum_{x \in supp(p)}u(x)p(x) \right) \delta_w.$$
\end{thm}
\begin{proof}
  Let $n$ denote $\abs{supp(p)}$, the cardinality of $supp(p)$. We proceed by induction on $n$. \\
  \textbf{Base Case}: Suppose $n = 1$. Then $supp(p) = \{ x_1, \}, x_1 \in X$. Then $p(x_1) = 1$. We know that $u(x)$ is defined so that $\delta_{x_1} \sim u(x_1)\delta_b + (1 - u(x_1))\delta_w$. Observe that
  \begin{align*}
    p &\sim \delta_{x_1} \\
    &= \delta_{x_1}p(x_1) \\
    &\sim (u(x_1)\delta_b + (1 - u(x_1))\delta_w)p(x_1), \text{ by (\ref{lemma 3}),}  \\
    &\sim u(x_1)p(x_1)\delta_b + (p(x_1) - u(x_1)p(x_1))\delta_w \\
    &\sim u(x_1)p(x_1)\delta_b + (1 - u(x_1)p(x_1))\delta_w,
  \end{align*}
  \textbf{Inductive Step}: Suppose that the claim holds for $n \leq k - 1$, and, for $p \in P$, $n = k$. Then $supp(p) = \{x_1, \ldots x_{k} \}$. So $\sum_{i = 1}^{k}p(x_i) = 1$ and $\sum_{i = 1}^{k-1}p(x_1) = 1 - p(x_k)$. Observe
  \begin{align*}
    p &\sim p(x_1)\delta_{x_1} + \cdots + p(x_{k-1})\delta_{x_{k-1}} + p(x_k)\delta_{d_k} \\
    &= (1 - p(x_k)) \left( \frac{p(x_1)}{1 - p(x_k)}\delta_{x_1} + \cdots + \frac{p(x_{k-1})}{1 - p(x_k)}\delta_{x_{k-1}} \right) + p(x_k)\delta_{x_k} \\
    &= (1 - p(x_k) \sum_{i = 1}^{k - 1} \frac{p(x_i)}{1 - p(x_k)}\delta_{x_i} + p(x_k)\delta_{x_k}.
  \end{align*}
  Let $q = \sum_{i = 1}^{k - 1} \frac{p(x_i)}{1 - p(x_k)}\delta_{x_i}$, where $q(x_i) = \frac{p(x_i)}{1 - p(x_k)}$. Then we have that $q \in P$ and $p \sim (1 - p(x_k))q + p(x_k)\delta_{d_k}$.
  By the inductive hypothesis, since $\abs{supp(q) = k-1}$, $q \sim \sum_{i = 1}^{k - 1}u(x_i)q(x_i)\delta_b + (1 - \sum_{i = 1}^{k - 1}u(x_i)q(x_i))\delta_w$. Then by (\ref{lemma 2}), $\delta_{x_k} \sim u(x_k)\delta_b + (1 - u(x_k))\delta_w$.
  Observe that
  \begin{align*}
    (1 - p(x_k))q + p(x_k)\delta_{x_k} &\sim (1 - p(x_k))\left(\sum_{i = 1}^{k - 1}u(x_i)q(x_i)\delta_b + \left(1 - \sum_{i = 1}^{k - 1}u(x_i)q(x_i)\right)\delta_w\right)  \\ &+ \ p(x_k)\delta_{x_k}, \text{ by (\ref{lemma 3}),} \\
    &\sim \left(\sum_{i = 1}^{k - 1}u(x_i)p(x_i)\delta_b + \left(1 - p(x_k) - \sum_{i = 1}^{k - 1}u(x_i)p(x_i)\right)\delta_w\right) +  p(x_k)\delta_{x_k} \\
    &\sim \left(\sum_{i = 1}^{k - 1}u(x_i)p(x_i)\delta_b + \left(1 - p(x_k) - \sum_{i = 1}^{k - 1}u(x_i)p(x_i)\right)\delta_w\right) \\ &+  p(x_k)(u(x_k)\delta_b + (1 - u(x_k))\delta_w), \text{ by (\ref{lemma 3}),} \\
    &\sim \sum_{i = 1}^{k - 1}u(x_i)p(x_i)\delta_b + p(x_k)u(x_k)\delta_b + \delta_w - \sum_{i = 1}^{k - 1}u(x_i)p(x_i)\delta_w - u(x_k)p(x_k)\delta_w \\
    &\sim \sum_{i = 1}^{k}u(x_i)p(x_i)\delta_b + \delta_w - \sum_{i = 1}^{k}u(x_i)p(x_i)\delta_w\\
    &\sim \sum_{i = 1}^{k}u(x_i)p(x_i)\delta_b + (1 - \sum_{i = 1}^{k}u(x_i)p(x_i))\delta_w.
  \end{align*}
\end{proof}
%
% Problem 1(e)
%
\begin{thm}
  The previous theorem holds even if there are no best and worst prizes.
\end{thm}
\begin{proof}
  Suppose there are two elements of $P$ that exhibit strict preference; otherwise, the result follows trivially. Let us denote these elements $b, w$. It is sufficient to provide a $u(\cdot)$ with the desired properties. Since $b \succ w$, let define $u(b) = 1$ and $u(w) = 0$. So for $x \in X$, either $\delta_x \succ \delta_b$, $\delta_b \succeq \delta_x \succeq \delta_w$, or $\delta_w \succ \delta_x$.
  \begin{enumerate}
    \item Suppose $\delta_x \succ \delta_b$. Since $\delta_x \succ \delta_b \succ \delta_w$, by (\ref{lemma 2}), $\delta_b \sim \alpha \delta_x + (1 - \alpha)\delta_w, \alpha \in (0,1)$. Let us propose that $u(x) = \frac{1}{\alpha}$. By (\ref{lemma 3}),
    \begin{align*}
      \frac{1}{\alpha}\delta_b + \left(1 - \frac{1}{\alpha}\right)\delta_w &\sim \frac{1}{\alpha}\alpha \delta_x + (1 - \alpha)\delta_w)\ + \left(1 - \frac{1}{\alpha}\right)\delta_w \\
      &= \delta_x + \frac{1}{\alpha}\delta_w - \delta_w + \delta_w - \frac{1}{\alpha}\delta_w \\
      &= \delta_x.
    \end{align*}
    \item Suppose $\delta_b \succeq \delta_x \succeq \delta_w$. By (\ref{lem1c.3}) and (\ref{lemma 2}) there exists a unique $\alpha \in [0,1]$ such that $\delta_x \sim \alpha \delta_b + (1 - \alpha)\delta_w$. Then $u(x) = \alpha$ is sufficient.
    \item Suppose $\delta_w \succ \delta_x$. Since $\delta_b \succ \delta_w \succ \delta_x$, by (\ref{lemma 2}), $\delta_w \sim \alpha \delta_b + (1 - \alpha)\delta_w$. Let us propose that $u(x) = \frac{-\alpha}{1 - \alpha}$. By (\ref{lemma 3}),
    \begin{align*}
      \left( 1 - \frac{-\alpha}{1 - \alpha}\right) \delta_w + \frac{-\alpha}{1 - \alpha}\delta_b &\sim \left( 1 - \frac{-\alpha}{1 - \alpha}\right) (\alpha \delta_b + (1 - \alpha)\delta_x) + \frac{-\alpha}{1 - \alpha}\delta_b \\
      &= (\alpha \delta_b + (1 - \alpha)\delta_x) + (\alpha \delta_b + (1 - \alpha)\delta_x)\frac{\alpha}{1 - \alpha} - \frac{\alpha}{1 - \alpha}\delta_b \\
      &= \alpha \delta_b + (1 - \alpha)\delta_x + \frac{\alpha^2}{1 - \alpha}\delta_b + \alpha\delta_x - \frac{\alpha}{1 - \alpha}\delta_b \\
      &= \frac{\alpha - \alpha^2}{1 - \alpha} \delta_b + \delta_x - \alpha\delta_x + \frac{\alpha^2}{1 - \alpha}\delta_b + \alpha\delta_x - \frac{\alpha}{1 - \alpha}\delta_b \\
      &= \delta_x.
    \end{align*}
  \end{enumerate}
\end{proof}
%
% Problem 1(f)
%
\begin{thm}
  If $u(\cdot)$ is an expected utility representation for $\succ$ on $X$ then $u(\cdot)$ is unique up to a positive linear transformation.
\end{thm}
\begin{proof}
  Suppose $u,v: X \rightarrow \RR$ both give expected utility representations for $\succ$ on $X$. For contradiction, suppose $v(\cdot) \not = au(\cdot) + b, a \in \RR^+, b \in \RR$. Suppose $x, y, z \in X$ such that $\delta_x \succ \delta_y \succ \delta_z$; otherwise, the result is trivial. Then $u(x) > u(y) > u(z)$ and $v(x) > v(y) > v(z)$.
  Then there exists a unique $\alpha \in (0,1)$ such that $u(y) = \alpha u(x) + (1 - \alpha)u(z)$. This implies $\delta_y \sim \alpha \delta_x + (1 - \alpha)\delta_z$. Since $v(\cdot)$ is a expected utility representation of $\succ$, $v(y) = \alpha v(x) + (1 - \alpha)v(z)$. Solving for $\alpha$, we find that $\frac{u(y) - u(z)}{u(x) - u(z)} = \alpha = \frac{v(y) - v(z)}{v(x) - v(z)}$.
  Observe that
  \begin{align*}
    &          & \frac{u(y) - u(z)}{u(x) - u(z)} &= \frac{v(y) - v(z)}{v(x) - v(z)} \\
    & \implies & (v(x) - v(z))(u(y) - u(z)) &= (u(x) - u(z))(v(y) - v(z)) \\
    & \implies & u(y)v(x) - u(y)v(z) - u(z)v(x) + u(z)v(z) &= u(x)v(y) - u(z)v(y) - u(x)v(z) + u(z)v(z) \\
    & \implies & u(y)v(x) - u(y)v(z) - u(z)v(x) &= u(x)v(y) - u(z)v(y) - u(x)v(z) \\
    & \implies & u(y)(v(x) - v(z)) - u(z)v(x) + u(x)v(z) &= (u(x) - u(z))v(y) \\
    & \implies & u(y)\left(\frac{v(x) - v(z)}{u(x) - u(z)}\right) - \frac{u(z)v(x) + u(x)v(z)}{u(x) - u(z)} &= v(y) \\
  \end{align*}
  Since $v(x) - v(z) > 0$ and $u(x) - u(z) > 0$, we have that $\frac{v(x) - v(z)}{u(x) - u(z)}$. $(\contra)$.
\end{proof}
%
% Problem 2
%
\section{}
\begin{thm}
  Suppose $p' \sim \frac{2}{3}\delta_{\$ 10}$ + $\frac{1}{3}\delta_{\$ 20}$ and $p \sim \frac{1}{3}\delta_{\$ 5} + \frac{5}{9}\delta_{\$ 15} + \frac{1}{9}\delta_{\$ 30}$.
  Then a risk averse expected utility maximizer will have $p' \succeq p$.
\end{thm}
Let us first prove a helpful lemma that generalizes (\ref{lem1c.2}).
\begin{lem}\label{lem 2.1}
  If $p \succeq q, r \succeq s, \alpha \in [0,1]$, then $\alpha p + (1 - \alpha)r \succeq \alpha q + (1 - \alpha)s$.
\end{lem}
\begin{proof}
  Observe if $\alpha = 1$ or $\alpha = 0$ then the result is trivial. Suppose $\alpha \in (0,1)$.
  \begin{enumerate}
    \item Suppose $p \succ q$. By the Substitution Axiom, $\alpha p + (1 - \alpha)r \succ \alpha q + (1 - \alpha)r$.
    \begin{enumerate}
      \item Suppose $r \succ s$. By the Substitution Axiom, $(1 - \alpha)r + \alpha q \succ (1 - \alpha)s + \alpha q$. Since $\succ$ is transitive, we have that $\alpha p + (1 - \alpha)r \succ (1 - \alpha)s + \alpha q$, implying $\alpha p + (1 - \alpha)r \succeq (1 - \alpha)s + \alpha q$.
      \item Suppose $r \sim s$. By (\ref{lemma 3}), $(1 - \alpha)r + \alpha q \sim (1 - \alpha)s + \alpha q$. By (\ref{lem1c.1}), $\alpha p + (1 - \alpha)r \succ (1 - \alpha)s + \alpha q$, implying $\alpha p + (1 - \alpha)r \succeq (1 - \alpha)s + \alpha q$.
    \end{enumerate}
    \item Suppose $p \sim q$. By (\ref{lemma 3}), $\alpha p + (1 - \alpha)r \sim \alpha q + (1 - \alpha)r$.
    \begin{enumerate}
      \item Suppose $r \succ s$. By the Substitution Axiom, $(1 - \alpha)r + \alpha q \succ (1 - \alpha)s + \alpha q$. By (\ref{lem1c.1}), $\alpha p + (1 - \alpha)r \succ (1 - \alpha)s + \alpha q$, implying $\alpha p + (1 - \alpha)r \succeq (1 - \alpha)s + \alpha q$.
      \item Suppose $r \sim s$. By (\ref{lemma 3}), $(1 - \alpha)r + \alpha q \sim (1 - \alpha)r + \alpha q$. Since $\sim$ is transitive, $\alpha p + (1 - \alpha)r \succ (1 - \alpha)s + \alpha q$, implying $\alpha p + (1 - \alpha)r \succeq (1 - \alpha)s + \alpha q$.
    \end{enumerate}
  \end{enumerate}
\end{proof}
\begin{proof}
  Since the agent is risk averse, the agent's utility function over dollar values has the following property: for $p \in P, \delta_{E[p]} \succeq p$. Let us define $q = \frac{1}{2}\delta_{\$ 5} + \frac{1}{2}\delta_{\$ 15}$ and $r = \frac{2}{3}\delta_{\$ 15} + \frac{1}{3}\delta_{\$ 30}$.
  Observe that $E[q] = \$ 10, E[r] = \$ 20$. Then $\delta_{E[q]} \succeq q$ and $\delta_{E[r]} \succeq r$, since the agent is risk averse. This implies that $\delta_{\$ 10} \succeq q$ and $\delta_{\$ 20} \succeq r$.
  By (\ref{lem 2.1}), $\frac{2}{3}\delta_{\$ 10} + \frac{1}{3}\delta_{\$ 20} \succeq \frac{2}{3}q + \frac{1}{3}r$.
  Observe that
  \begin{align*}
    p' &\sim \frac{2}{3}\delta_{\$ 10} + \frac{1}{3}\delta_{\$ 20} \\
    &\succeq \frac{2}{3}q + \frac{1}{3}r \\
    &= \frac{2}{3}(\frac{1}{2}\delta_{\$ 5} + \frac{1}{2}\delta_{\$ 15}) + \frac{1}{3}(\frac{2}{3}\delta_{\$ 15} + \frac{1}{3}\delta_{\$ 30}) \\
    &= \frac{1}{3}\delta_{\$ 5} + \frac{5}{9}\delta_{\$ 15} + \frac{1}{9}\delta_{\$ 30} \\
    &\sim p.
  \end{align*}
  By transitivity of $\succeq$, $p' \succeq p$.
\end{proof}
%
% Problem 3(a)
%
\section{}
\begin{prob}
Suppose an agent in a two good world has the ordinal utility function $u(x_1, x_2) = x_1 + x_2$ and income $Y > 0$. Let the agent's von Neumann-Morgenstern utility function be given by $f(u(x))$, where $f$ is a strictly increasing function. Suppose prices are not fixed and $p = (3,1), p' = (1,3)$ are two price vectors that occur with probability $\frac{1}{2}$. Show that this agent prefers risky prices to centain prices.
\end{prob}
\begin{proof}[Solution]
  Let $Y$ be fixed. For each price vector $p$, let us construct the optimal given income, i.e., construct the indirect utility function $\nu(p, Y)$. Define $i$ as the index of the value satisfying $\min \{ p_1, p_2 \}$.
  Then the optimal bundle at prices $p$ contains only $\frac{Y}{p_i}$ of good $x_i$. So $\nu(p, Y) = \frac{Y}{p_i}$. Likewise, at prices $p'$ the optimal bundle given by $\nu(p', Y) = \frac{Y}{p'_i}$. The intuition is that the agent is buying only the cheaper good and as much as the agent can afford. Let $\rho(p)$ denote the probability that prices $p$ obtain. The expected utility at risky prices is given by
  \begin{align}
    \nu(p, Y)\rho(p) + \nu(p', Y)\rho(p') &= \frac{Y}{p_i}\frac{1}{2} + \frac{Y}{p'_i}\frac{1}{2} \\
    &= \frac{Y}{2} + \frac{Y}{2} \\
    &= Y.
  \end{align}
  On the other hand, let $q = \frac{1}{2}p + \frac{1}{2}p' = (2,2)$. The $\nu(q, Y) = \frac{Y}{q_1} = \frac{Y}{q_2}$, since the price is the same for each good. So the expected utility is given by $\frac{Y}{2}$. Since $f$ is strictly increaseing, $f(Y) > f(\frac{Y}{2})$. Hence, risky prices are preferred to certain prices in this case.
\end{proof}
%
% Problem 3(b)
%
\begin{prob}
  Coming soon!
\end{prob}
%
% Problem 4
%
\section{}
\begin{prob}
  Suppose an agent has a concave utility function for net income that is not necessairly differentiable. What changes does this cause for the analysis of insurance demand?
\end{prob}
\begin{proof}[Solution]
  This is a vague question where the obvious answers are more mathematically sophisticated than the general level of the text. Recall that a concave function may have countably many points of nondifferentiability. As a result, the derivative may not be everywhere defined. Thus, the methods in the text cannot be applied straightforwardly and we must move on to generalizations of the derivative or non-derivative methods to proceed. This seems to be a reasonable answer though it's possible that I have missed the point of the exercise.
\end{proof}
%
% Problem 5
%
\section{}
Coming soon!
%
% Problem 6
%
\section{}
Suppose an agent has $\$ W$ to invest between a sure thing and a risky asset. The sure thing has a return of $\$ r > 1$. The risky asset returns a random ammount $\theta$, where $\theta$ is a simple lottery on $(0, \infty)$ with distribution $\pi$. Suppose $E \theta > r$ but $\pi (\theta < r) > 0$. Suppose the agent has the ability to short sell, i.e., the agent's utility function is defined for negative income.
The agent has a constant coefficient of absolute risk aversion $\lambda > 0$, i.e., the agent allocates his funds to maximize the expectation of $-e^{- \lambda Y}$ where $Y$ is the net payoff from the constructed portfolio. We write $\alpha(W, \lambda)$ for the optimal amount to invest in the risk asset.
\begin{thm}
  For all $W, \lambda$, $\alpha(W, \lambda) < \infty$.
\end{thm}
\begin{proof}
  Suppose for $W, \lambda$, $\alpha(W, \lambda)$ is infinite, i.e., $\arg \max_\alpha E \left[-e^{- \lambda Y} \right]$ has no solution. Observe that
  \begin{align*}
    E \left[-e^{- \lambda Y} \right] &= E \left[-e^{- \lambda ((W - \alpha)r + \alpha \theta)} \right] \\
    &= E \left[-e^{- \lambda (Wr + \alpha (\theta - r))} \right] \\
    &= \sum_{x \in supp(\pi)} -e^{- \lambda (Wr + \alpha (x - r))} \pi(x).
  \end{align*}
  Since the maximum is not defined with $\alpha \rightarrow \infty$, consider $\lim_{\alpha \rightarrow \infty} \sum_{x \in supp(\pi)} -e^{- \lambda (Wr + \alpha (x - r))}$. By assumption, there exists $x^* \in supp(\pi)$ such that $x^* < r$.
  For this term in the summation, $\lim_{\alpha \rightarrow \infty} -e^{- \lambda(Wr + \alpha(x - r))} = \lim_{\alpha \rightarrow \infty} -e^{\alpha} = - \infty$.
  Then this expected value equals $- \infty$, if it is defined. But $\alpha = 0$ yields
  \begin{align*}
    \sum_{x \in supp(\pi)} -e^{- \lambda (Wr + \alpha (x - r))} \pi(x) &= \sum_{x \in supp(\pi)} -e^{- \lambda Wr } \pi(x) \\
    &= \sum_{x \in supp(\pi)} -e^{- \lambda Wr } \pi(x) \\
    &= -e^{- \lambda Wr } \\
    &> - \infty.
  \end{align*}
  So the supposed $\alpha$ does not maximize the expected value. $(\contra)$.
\end{proof}
\begin{thm}
  $\alpha(W, \lambda)$ is independent of $W$.
\end{thm}
\begin{proof}
  Let $\lambda > 0$ be given and $W = \beta Z, \beta \in \RR^+$. Let $\gamma_W = \alpha(W, \lambda)$ and $\gamma_Z = \alpha(Z, \lambda)$. Define the functions $f(\alpha) = E \left[-e^{- \lambda ((W - \alpha)r + \alpha \theta)} \right]$ and $g(\alpha) = E \left[-e^{- \lambda ((Z - \alpha)r + \alpha \theta)} \right]$.
  Then $f(\gamma_W) \geq f(\alpha), \alpha \in [0, \infty )$ and $g(\gamma_Z) \geq g(\alpha), \alpha \in [0, \infty )$.
  Let $\eta = e^{\lambda (\beta - 1)Zr}$. Then $\eta > 0$. Observe that
  \begin{align*}
    &       & g(\gamma_Z) &\geq g(\alpha) \\
    & \iff  & g(\gamma_Z)\eta &\geq g(\alpha)\eta \\
    & \iff  & \sum_{x \in supp(\pi)} -e^{- \lambda (Zr + \gamma_Z (x - r))} \pi(x) \eta &\geq \sum_{x \in supp(\pi)} -e^{- \lambda (Zr + \alpha (x - r))} \pi(x) \eta \\
    & \iff  & \sum_{x \in supp(\pi)} -e^{- \lambda (Zr + \gamma_Z (x - r))}  \pi(x) e^{\lambda (\beta - 1)Zr} &\geq \sum_{x \in supp(\pi)} -e^{- \lambda (Zr + \alpha (x - r))} \pi(x) e^{\lambda (\beta - 1)Zr} \\
    & \iff  & \sum_{x \in supp(\pi)} -e^{- \lambda (Zr + \gamma_Z (x - r)) + \lambda (\beta - 1)Zr}  \pi(x) &\geq \sum_{x \in supp(\pi)} -e^{- \lambda (Zr + \alpha (x - r)) + \lambda (\beta - 1)Zr} \pi(x) \\
    & \iff  & \sum_{x \in supp(\pi)} -e^{- \lambda (\beta Zr + \gamma_Z (x - r))}  \pi(x) &\geq \sum_{x \in supp(\pi)} -e^{- \lambda (\beta Zr + \alpha (x - r))} \pi(x) \\
    & \iff  & \sum_{x \in supp(\pi)} -e^{- \lambda (Wr + \gamma_Z (x - r))}  \pi(x) &\geq \sum_{x \in supp(\pi)} -e^{- \lambda (Wr + \alpha (x - r))} \pi(x) \\
    & \iff  & f(\gamma_Z) &\geq f(\alpha).
  \end{align*}
  Then $\gamma_Z$ maximizes $f$.
\end{proof}
\begin{thm}
  $\alpha(W, \lambda)$ is nonincreasing in $\lambda$.
\end{thm}
\begin{proof}
Coming soon!
\end{proof}
%
% Problem 7
%
\section{}
\begin{thm}
  Consider the consumer's problem with one risky asset. If the consumer's vNM utility function is strictly concave then the consumer's problem has a unique solution, if it has a solution.
\end{thm}
\begin{proof}
  The consumer's problem with one risky asset is $$\mathcal{P} = \max_\alpha \sum_{\theta \in supp(\pi)} \nu(\alpha(\theta - r) + rW)\pi(\theta)$$.
  Suppose $\alpha_1, \alpha_2$ are both solutions to $\mathcal{P}$. Let $\gamma \in (0,1)$. Then for $\theta \in supp(\pi)$,
  \begin{align*}
    \nu((\gamma \alpha_1 + (1 - \gamma)\alpha_2)(\theta - r) + rW) &= \nu(\gamma \alpha_1(\theta - r) + \gamma rW + (1 - \gamma)\alpha_2(\theta - r) + (1 - \gamma)rW) \\
    &= \nu(\gamma(\alpha_1(\theta - r) + rW) + (1 - \gamma)(\alpha_2(\theta - r) + rW)) \\
    &> \gamma \nu(\alpha_1(\theta - r) + rw) + (1 - \gamma)\nu(\alpha_2(\theta - r) + rW).
  \end{align*}
  The strict inequality in the third line follows from strict concavity of $\nu$. Observe that
  \begin{align*}
    & \sum_{\theta \in supp(\pi)} \nu((\gamma \alpha_1 + (1 - \gamma)\alpha_2)(\theta - r) + rW)\pi(\theta) \\
    &> \sum_{\theta \in supp(\pi)} (\gamma \nu(\alpha_1(\theta - r) + rw) + (1 - \gamma)\nu(\alpha_2(\theta - r) + rW)) \pi(\theta) \\
    &= \sum_{\theta \in supp(\pi)} \gamma \nu(\alpha_1(\theta - r) + rw)\pi(\theta) + \sum_{\theta \in supp(\pi)} (1 - \gamma)\nu(\alpha_2(\theta - r) + rW)\pi(\theta) \\
    &= \gamma \sum_{\theta \in supp(\pi)} \nu(\alpha_1(\theta - r) + rw)\pi(\theta) + (1 - \gamma)\sum_{\theta \in supp(\pi)} \nu(\alpha_2(\theta - r) + rW)\pi(\theta) \\
    &= \gamma \sum_{\theta \in supp(\pi)} \nu(\alpha_1(\theta - r) + rw)\pi(\theta) + (1 - \gamma)\sum_{\theta \in supp(\pi)} \nu(\alpha_1(\theta - r) + rW)\pi(\theta) \\
    &= \sum_{\theta \in supp(\pi)} \nu(\alpha_1(\theta - r) + rw)\pi(\theta).
  \end{align*}
  Observe that the equality in the fourth line follows since $\alpha_1, \alpha_2$ are both solutions to $\mathcal{P}$. $(\contra)$.
\end{proof}
%
% Problem 8
%
\section{}
\begin{thm}
  It is possible for an asset to have an expected return less than $r$ and stil be demanded and an expected return greater than $r$ and not be demanded, in the context of multiple assets.
\end{thm}
\begin{proof}
  \begin{enumerate}
    \item Set $r = 2$. Suppose $(\theta_1, \theta_2) = (1, 5)$ with probability $\frac{1}{10}$ and $(\theta_1, \theta_2) = (5, 1)$ with probability $\frac{9}{10}$.
    Then the expected return of $\theta_1$ is $1 \left( \frac{1}{10}\right) + 5 \left( \frac{9}{10}\right) = \frac{46}{10} = 4.6$
    and the expecte return of $\theta_2$ is $5 \left( \frac{1}{10}\right) + 1 \left( \frac{9}{10}\right) = \frac{14}{10} = 1.4 < r$. Observe that
    \begin{align*}
      & \nu(1(5-2) + 1(1-2) + 2W) + \nu(1(1-2) + 1(5-2) + 2W) \\
      &= \nu(2 + 2W) + \nu(2 + 2W) \\
      &> \nu(2W) + \nu(2W) \\
      &= \nu(0(3) + 0(-1) + 2W) + \nu(0(-1) + 0(3) + 2W).
    \end{align*}
    Hence, $\theta_2$ is demanded some.
    \item Suppose $\theta_1 > \theta_2$, that $\theta_2$ is demanded at $\alpha_2 > 0$, and the expected return of $\theta_2$ is greater than $r$. Observe that
    \begin{align*}
      \nu(\alpha_1 (\theta_1 - r) + \alpha_2 (\theta_2 - r) + rW) &= \nu(\alpha_1 \theta_1 + \alpha_2 \theta_2 + r(W - (\alpha_1 + \alpha_2))) \\
      &< \nu(\alpha_1 \theta_1 + \alpha_2 \theta_1 + r(W - (\alpha_1 + \alpha_2))) \\
      &= \nu((\alpha_1 + \alpha_2) \theta_1 + r(W - (\alpha_1 + \alpha_2))) \\
    \end{align*}
    Any allocation to $\theta_2$ is better allocated to $\theta_1$. So $\theta_2$ has a return greater than the risk free rate but none the less not demanded.
  \end{enumerate}
\end{proof}
%
% Problem 9
%
\section{}
\begin{thm}
  If the returns for two risky assets are statistically independent, then the results obtained in Chapter 3.3 generalize.
\end{thm}
\begin{proof}[Solution]
  This is not difficult just tedious. Observe that since the two assets are statistically independent, then $\pi(\theta) = \pi(\theta_1, \theta_2) = \pi_1(\theta_1)\pi_2(\theta_2)$, where $\pi_1$ is the marginal distribution of $\pi$ with respect to $\theta_1$ and similarly for $\pi_2$.
  Then define $\theta_m = \min \{ \theta_1, \theta_2 \}$. We may then show the results for this reduced single risky asset.
\end{proof}
%
% Problem 10
%
\section{}
\begin{thm}
  If we have an \emph{additive across states representation} then for any strictly positive probability distribution $\pi$ on $S$, there is a state-dependent utility function $u: X \times S \rightarrow \RR$ such that subjective state-dependent expectied utility, using this probability distribution $\pi$ and utility function $u$ represents $\succ$.
\end{thm}
\begin{proof}
  Suppose $h \succ h'$. Then $\sum_{s \in S}u'(h(s),s)\pi'(s) > \sum_{s \in S}u'(h'(s),s)\pi'(s)$. Let $\pi$ be a probability distribution on $S$. Define $u(x, s) = \frac{u'(x, s)\pi'(s)}{\pi(s)}$. Observe that
  \begin{align*}
    \sum_{s \in S} u(h(s), s) \pi(s) &= \sum_{s \in S} u'(h(s), s)\pi'(s) \\
    &> \sum_{s \in S} u'(h'(s), s)\pi'(s) \\
    &= \sum_{s \in S} u(h'(s), s) \pi(s).
  \end{align*}
\end{proof}
%
% Problem 10
%
\section{}

\end{document}
